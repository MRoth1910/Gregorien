% !TEX TS-program = lualatexmk
% !TEX parameter =  --shell-escape

\documentclass[11pt]{book}
\input{commonheaders_vespers_people}

\begin{document}

\thispagestyle{empty}

\bigtitle{Adoration and Benediction on Sundays \textit{per annum} without Vespers.}

%{\centering{\includegraphics[width=100mm,height=500mm,keepaspectratio]{O_salutaris_Hostia_Duguet_melody}\par}}
{\centering{\includegraphics[scale=0.75,keepaspectratio]{O_salutaris_Hostia_Duguet_melody}\par}}

%{\centering{\includegraphics{O_salutaris_Hostia_Duguet_melody}\par}}

\begin{multicols}{2}
\raggedcolumns
\begin{otherlanguage}{english}
O saving victim, opening wide\\
The gate of heaven to man below,\\
Our foes press on from every side,\\
Thine aid supply, thy strength bestow.

To thy great name be endless praise,\\
Immortal Godhead, One in Three;\\
O grant us endless length of days\\
In our true native land, with thee.
Amen.
\end{otherlanguage}
\end{multicols}

\rubrique{Devotions are said according to the month.}

\begin{otherlanguage}{english}
{\centering{Prayer Before Benediction on Sundays.\kern 0.05em\footnote{\raggedright{Prayer adapted from Nicholas Stephen Cardinal Wiseman, Archbishop of\kern 0.25em West\-min\-ster (1802–1865).}}}\par}

\rubrique{This is said on all Sundays except for the second of each month.}

\rubrique{All kneel; the Priest begins:}

\lettrine{O}{} Blessed Virgin Mary, (\textit{all join}) Mother of God | and our most gentle Queen and Mother, | look down in mercy upon us all | who greatly hope and trust in thee. | By thee it was that Jesus | our Savior and our Hope | was given unto the world; | and He has given thee to us | that we might hope still more.

Plead for us, thy children, | whom thou didst receive and accept | at the foot of the Cross, | O sorrowful Mother. | Intercede for our separated brethren, | that with us, | in the one true fold, | they may be united to the supreme Shepherd, | the Vicar of thy Son.

Pray for us all, dear Mother | that by faith | fruitful in good works, | we may all deserve to see and praise God, | together with thee, | in our heavenly home. Amen.

\rubrique{On the Second Sunday of each month, instead of the prayer \normaltext{O Blessed Virgin Mary,} the following prayer is said, all kneeling.}

\rubrique{The priest says:}

\lettrine{O}{} merciful God, let the glorious intercession of Thy saints assist us; above all, the most blessèd Virgin Mary, the Immaculate Mother of Thine only-begotten Son, and Thy holy apostles, Peter and Paul, through whose patronage we humbly commend this land. Be mindful of our fathers, Thy glorious bishop John Neumann; of Junípero and Damian, Thy priests. Remember our holy martyrs who shed their blood for Christ: Isaac, René and Jean. Remember those holy virgins and widows: Frances, Rose-Philippine, Katherine, Théodore, Marianne, Kateri, and Elizabeth Ann; and all those holy men and women who made this country illustrious by their glorious merits and virtues. Let not thy memory perish before Thee, O Lord, but let their supplication enter daily into Thy sight; and do Thou, who didst so often spare Thy sinful people for the sake of Abraham, Isaac, and Jacob, now, also moved by the prayers of our fathers, brothers, and sisters reigning with Thee, have mercy up on us, save Thy people and bless Thine inheritance; and suffer not those souls to perish which Thy Son hath redeemed with His own most Precious Blood: Who liveth and reignest with Thee, world without end. ℟. Amen.
%\smallskip

Let us pray.

\lettrine{O}{} most loving Lord Jesus, Who, when Thou wert hanging on the Cross, didst commend us all in the person of Thy disciple John to Thy most sweet Mother, that we might find in her our refuge, our solace and our hope; look graciously upon our beloved land, and on those who are bereaved of so powerful a patronage; that acknowledging the dignity of this Holy Virgin, they may honor and venerate her with all affection of devotion, and own her as Queen and Mother. May her sweet name be listed by little ones, and linger on the lips of the agèd and dying; and may it be invoked by the afflicted, and hymned by the joyful; that this Star of the Sea being their protection and their guide, all may come to the harbor of eternal salvation. Who liveth and reignest with Thee, world without end. ℟. Amen.

\rubrique{Then follows Benediction of the Most Blessed Sacrament of the Eucharist, in the usual way.}
\end{otherlanguage}

%{\centering{\includegraphics[width=100mm,height=500mm,keepaspectratio]{Tantum_Ergo_Sacramentum_Wade_melody}\par}}

{\centering{\includegraphics[scale=0.75,keepaspectratio]{Tantum_Ergo_Sacramentum_Wade_melody}\par}}
\begin{multicols}{2}
\raggedcolumns
\begin{otherlanguage}{english}
Down in adoration falling,\\
Lo! the sacred Host we hail;\\
Lo! o'er ancient forms departing,\\
Newer rites of grace prevail;\\
Faith for all defects supplying,\\
Where the feeble senses fail.

To the everlasting Father,\\
And the Son who reigns on high,\\
With the Holy Ghost proceeding\\
Forth from each eternally,\\
Be salvation, honour, blessing,\\
Might, and endless majesty.
Amen.
\end{otherlanguage}
\end{multicols}

   \begin{paracol}{2}
%\sloppy
\noindent \vv Panem de cælo præstitisti eis.
\tpalleluia{}\footnote{Atque per octavam Corporis Christi. \textit{As well as throughout the octave of Corpus Christi.}}

\noindent \rr Omne dele\-ctaméntum in se habéntem.
\tpalleluia{}

%\newpage

  Orémus.
  
  \lettrine{D}{e}us, qui nobis sub Sacraménto mirábili passiónis tuæ memóriam reliquísti:~† tríbue, quǽsumus, ita nos córporis, et sánguinis tui sacra mystéria venerári;~* ut redemptiónis tuæ fructum in nobis júgiter sentiámus. Qui vivis et regnas in sǽcula sæculórum. 

  \switchcolumn
\begin{otherlanguage}{english}
\noindent \vv Thou hast given them bread from heaven.
(\textit{P.T.} alleluia.)

\noindent \rr Having all sweetness within it.
(\textit{P.T.} alleluia.)
%\newpage

\noindent Let us pray. O God, who under a wonderful Sacrament hast left us a memorial of Thy Passion; grant us, we beseech Thee, so to reverence the sacred mysteries of Thy Body and Blood, that we may ever feel within ourselves the fruit of Thy Redemption. Who livest and reignest forever and ever. Amen.
\end{otherlanguage}\end{paracol}

\begin{multicols}{2}
\raggedcolumns
\begin{otherlanguage}{english}

Blessed be God. 

Blessed be His Holy Name. 

Blessed be Jesus Christ, true God and true Man.
 
Blessed be the Name of Jesus.

Blessed be His Most Sacred Heart.

Blessed be His Most Precious Blood.

Blessed be Jesus in the Most Holy Sacrament of the Altar.

Blessed be the Holy Spirit, the Paraclete.

Blessed be the great Mother of God, Mary most Holy.

Blessed be her Holy and Immaculate Conception.

\rubrique{The following invocation is said thrice.}

Blessed be her Glorious Assumption.

Blessed be the name of Mary, Virgin and Mother.

Blessed be Saint Joseph, her most chaste spouse.

Blessed be God in His Angels and in His Saints.

\end{otherlanguage}
\end{multicols}

\rubrique{These are sung alternating between a cantor and the whole congregation, which also sings \normaltext{Fiat, Fiat} at the end.}

\gscore[]{}{laudes_divinae_no_breaks}

%{\centering{\includegraphics[width=100mm,height=500mm,keepaspectratio]{holy_god_we_praise_thy_name_melody}\par}}
{\centering{\includegraphics[scale=0.75,keepaspectratio]{holy_god_we_praise_thy_name_melody}\par}}
\rubrique{Or}

%{\centering{\includegraphics[width=100mm,height=500mm,keepaspectratio]{sweet_sacrament_melody}\par}}
{\centering{\includegraphics[scale=0.75,keepaspectratio]{sweet_sacrament_melody}\par}}

\rubrique{In June:}

\begin{otherlanguage}{english}\rubrique{The cantors intone until the half bar, then the people are welcome to sing the rest of the antiphon and the repetitions.}\end{otherlanguage}

\rubrique{Melody adapted, probably by Dom Joseph Pothier, o.s.b.}

\gscore[]{1.}{va--cor_jesu--solesmes}
\begin{otherlanguage}{english}Most Sacred Heart of Jesus, have mercy on us.
\end{otherlanguage}


\enddocument