% !TEX TS-program = lualatexmk
% !TEX parameter =  --shell-escape

\documentclass[11pt]{article}
\input{commonheaders_vespers_people}

\begin{document}

\thispagestyle{empty}

\bigtitle{Exposition and Benediction on Weekdays of Paschal Time.}

\rubrique{The cantors intone until the first barline.}

\rubrique{\normaltext{O Salutáris III, Chants of the Church,} 1956. Melody of the hymn at Lauds of Pentecost.}

\gscore[]{1.}{hy--o_salutaris_hostia_pentecost_lauds_solesmes}
\begin{multicols}{2}
\raggedcolumns
\begin{otherlanguage}{english}
O saving victim, opening wide\\
The gate of heaven to man below,\\
Our foes press on from every side,\\
Thine aid supply, thy strength bestow.

To thy great name be endless praise,\\
Immortal Godhead, One in Three;\\
O grant us endless length of days\\
In our true native land, with thee.
Amen.
\end{otherlanguage}
\end{multicols}

\begin{otherlanguage}{english}{\centering{Prayer Before Benediction on Thursdays.}\par}

\rubrique{The priest says this prayer.}

\lettrine{O}{} most sweet Jesus, who didst come into this world to enrich all souls with the life of Thy grace, and in order to preserve and foster that life within them, dost give Thyself day by day in the most adorable Sacrament of the Eucharist, as a saving remedy to heal their infirmities and a divine food to support their weakness: we humbly beseech Thee graciously to pour forth Thy Holy Spirit upon them all; so that, filled therewith, such as are defiled with deadly stain may return to Thee and recover the life of grace which they have forfeited by sin, and such as by the gift of Thy mercy already cleave to Thee may, as to each is given, approach every day Thy heavenly Banquet with true devotion. Strengthened thus, may they find herein an antidote to daily venial faults and a nourishing of the life of grace; that so being ever more and more cleansed, they may come to enjoy everlasting happiness in Heaven. ℟. Amen.

\rubrique{Then follows the prayer for the Sovereign Pontiff:}\footnote{\raggedright{Partialis indulgentia conceditur christifideli qui,…spiritu filialis devotionis, pro Summo Pontific aliquam prece legitime adprobatam pi recitaverit \textit{(e.g. Orémus pro Pontifice)} (Enchiridion Indulgentiarum, nº 25, 1º)}}

\noindent \vv Let us pray for our Pontiff, \textit{N.}

\noindent \rr The Lord preserve him, and give him life, | and make him blessèd upon the earth, | and deliver him up not up to the will of his enemies.

\noindent \vv Our Father…

\noindent \rr Give us this day…

\noindent \vv Hail Mary…

\noindent \rr Holy Mary…

\noindent \vv  Glory be…

\noindent \rr As it was…world without end. Amen.

\rubrique{Priest:}

\lettrine{A}{l}mighty, everlasting God, have mercy upon Thy servant, \textit{N.,} our Sovereign Pontiff, and direct him, according to Thy clemency, into the way of everlasting salvation; that by Thy grace he may desire those things that are pleasing to Thee, and perform them with all his strength. ℟. Amen.

\rubrique{On First Friday, the Sacred Heart devotions are said instead. Then follows Benediction of the Most Blessed Sacrament, as usual.}

\end{otherlanguage}

\rubrique{\normaltext{Tantum ergo IX} from \normaltext{Cantus Selecti,} 1957; melody of \normaltext{Urbs beáta Jerúsalem.}}
\gscore[]{4.}{hy--tantum_ergo_ix--solesmes_1957}
\begin{multicols}{2}
\raggedcolumns
\begin{otherlanguage}{english}
Down in adoration falling,\\
Lo! the sacred Host we hail;\\
Lo! o'er ancient forms departing,\\
Newer rites of grace prevail;\\
Faith for all defects supplying,\\
Where the feeble senses fail.

To the everlasting Father,\\
And the Son who reigns on high,\\
With the Holy Ghost proceeding\\
Forth from each eternally,\\
Be salvation, honour, blessing,\\
Might, and endless majesty.
Amen.
\end{otherlanguage}
\end{multicols}

   \begin{paracol}{2}
   
\noindent \vv Panem de cælo præstitisti eis, allelúia.

\noindent \rr Omne dele\-ctaméntum in se habéntem, allelúia.
  
  Orémus.
  
  \lettrine{D}{e}us, qui nobis sub Sacraménto mirábili passiónis tuæ memóriam reliquísti:~† tríbue, quǽsumus, ita nos córporis, et sánguinis tui sacra mystéria venerári;~* ut redemptiónis tuæ fructum in nobis júgiter sentiámus. Qui vivis et regnas in sǽcula sæculórum. 
  
  \rr Amen.
  \switchcolumn
\begin{otherlanguage}{english}
\noindent \vv Thou hast given them bread from heaven, alleluia.

\noindent \rr Having all sweetness within it, alleluia.

\noindent Let us pray. O God, who under a wonderful Sacrament hast left us a memorial of Thy Passion; grant us, we beseech Thee, so to reverence the sacred mysteries of Thy Body and Blood, that we may ever feel within ourselves the fruit of Thy Redemption. Who livest and reignest forever and ever. Amen.

\end{otherlanguage}
\end{paracol}

\begin{multicols}{2}
\raggedcolumns
\begin{otherlanguage}{english}

Blessed be God. 

Blessed be His Holy Name. 

Blessed be Jesus Christ, true God and true Man.
 
Blessed be the Name of Jesus.

Blessed be His Most Sacred Heart.

Blessed be His Most Precious Blood.

Blessed be Jesus in the Most Holy Sacrament of the Altar.

Blessed be the Holy Spirit, the Paraclete.

Blessed be the great Mother of God, Mary most Holy.

Blessed be her Holy and Immaculate Conception.

\rubrique{The following invocation is said thrice.}

Blessed be her Glorious Assumption.

Blessed be the name of Mary, Virgin and Mother.

Blessed be Saint Joseph, her most chaste spouse.

Blessed be God in His Angels and in His Saints.

\end{otherlanguage}
\end{multicols}

\rubrique{These are sung alternating between a cantor and the whole congregation, which also sings \normaltext{Fiat, Fiat} at the end.}

\gscore[]{}{laudes_divinae_no_breaks}

\rubrique{The cantors sing the odd verses, all others even.}

\rubrique{Psalm tone for psalms without an antiphon in Paschal Time from the \normaltext{Antiphonale Romanum.}}

\gscore[]{}{ps_116_tp}

\begin{otherlanguage}{english}\noindent Ps. 116: Praise the Lord, all ye nations: praise him, all ye people. For his mercy is confirmed upon us: and the truth of the Lord remaineth forever. Glory be to the Father, and to the Son, and to the Holy Ghost. As it was in the beginning, is now, and ever shall be, world without end. Amen.\end{otherlanguage}

\rubrique{On First Friday, after Communion:}

\rubrique{Melody adapted, probably by Dom Joseph Pothier, o.s.b.}

\begin{otherlanguage}{english}\rubrique{The cantors intone until the half bar, then the people are welcome to sing the rest of the antiphon and the repetitions.}\end{otherlanguage}

\gscore[]{1.}{va--cor_jesu--solesmes}

%\noindent
\begin{otherlanguage}{english}Most Sacred Heart of Jesus, have mercy on us.
\end{otherlanguage}

\end{document}