% !TEX TS-program = lualatexmk
% !TEX parameter =  --shell-escape

\documentclass[11pt]{book}
%%%%%%%%%%%%%%% STANDARD PACKAGES %%%%%%%%%%%%%%%

%% for personal reasons, the commands are a mix of French and English names. Feel free to choose one or the other if you make use of them. This file is subject to change later on.

\usepackage{geometry}
\usepackage{fontspec}
\usepackage[english,ecclesiasticlatin.usej,activeacute]{babel}
     \babelprovide[hyphenrules=latin]{ecclesiasticlatin}
%          \usepackage{xspace} %% better to use {} after certain macros (or even within definitions) 
     \usepackage{multicol}
\usepackage{paracol}
\footnotelayout{m} %% this allows merged footnote if text is in parallel columns (mostly translations)
\usepackage{fancyhdr}
\usepackage{needspace} %%pour réserver de l'espace en bas de page ; ça évite des séparations entre les titres et la partition ou le texte qui les suivent.
\usepackage[compact]{titlesec}
\usepackage{xcolor}
%\usepackage{xstring}
\usepackage{enumitem} %% nécessaire pour les textes des psaumes.
\usepackage{lettrine}
\usepackage{tocloft}%%to fix toc title. titlesec companion packages could also work.
\usepackage{indentfirst}%% to allow for \section with indented 1st paragraphs in good Franco-Latin style (LaTeX default is to suppress this. Package sets  \@afterindentfalse to (always) true.
\usepackage[savepos]{zref}
\usepackage{lettrine}
\usepackage{hyperref}
\usepackage{refcount}


%%%%%%%%%%%%%%% HYPHENATION AND TYPOGRAPHICAL CONVENTIONS %%%%%%%%%%%%%%

%% page settings %%%%
\sloppy %% to avoid overfull hboxes

%\setlength{\textwidth}{5.75in}

\geometry{letterpaper}
\geometry{bindingoffset=0.5in,
inner=1in,
outer=1.35in,
top=1.25in,
bottom=1.25in,
headsep=0.75in
} %%

%% General scale of all graphical elements. Also borrowed from MB
%% Values different from 1 are largely untested.
%% Used in those commands (e.g. everything FontSpec) that use a scale parameter.
\newcommand{\customscale}{1}

\AtBeginDocument{\setlength{\parindent}{1em}} %% should set 1em to EB Garamond not Computer Modern.

%%%%%%%%%%%%%%% GREGORIO CONFIG %%%%%%%%%%%%%%%

\usepackage[autocompile]{gregoriotex}

%% sets * and † to font called via fontspec
\def\GreStar{*}
\def\GreDagger{†}

%% should prevent flat turning custos into a clef, as Solesmes  uses an altered custos in only one chant per Élie Roux.
\gresetcustosalteration{invisible}

%produces a score with a smaller initial (default is 40 pt) and annotations like in the Solesmes books; the 1st is optional and can be omitted as needed.
 \NewDocumentCommand{\gscore}{O{}mm}{%
 \greannotation{#1}%
\greannotation{#2}%
\grechangestyle{initial}{\fontsize{28}{28}\selectfont}%
 \gregorioscore{partitions/#3}}
 
 
  \NewDocumentCommand{\lectionscore}{O{}m}{%
   \grecommentary{#1}%
  \grechangestyle{initial}{\fontsize{36}{36}\selectfont}%
  \gregorioscore{partitions/#2}}
 
 %% default should be 40, if not using, annotations need to be turned off in gabc. For the first antiphon of the office, Gospel canticle etc. (initial is both to refer to first and the large illustrated initial, used sometimes with an annotation, cf. Christus factus est of Tenebrae in the ritus monasticus.
  \NewDocumentCommand{\initialscore}{O{}m}{%
      \grechangedim{initialraise}{0.5cm}{scalable}%
  \greannotation{#1}%
   \grechangestyle{initial}{\fontspec{EB Garamond Initials}\fontsize{45}{45}\selectfont}%
 \gregorioscore{partitions/#2}
   \grechangedim{initialraise}{0cm}{scalable}}

%% score with no initial, e.g psalms, common tones, etc.
\newcommand{\smallscore}[2][y]{
  \gresetinitiallines{0}
  \gregorioscore{partitions/#2} %% à remplacer avec NewDocumentCommand ; les arguments ne marchent pas correctement … que fait le [y] ?
  \gresetinitiallines{1}
}


 \NewDocumentCommand{\MagAnnotation}{}{%
 \grechangedim{annotationseparation}{-0.01cm}{fixed}%
 } %% pour les antiennes où le chiffre convient mieux aux miniscules (parfois 1, 2, 4, 7…) ; il n'est pas nécessaire pour 6 ou 8.
%\gresetheadercapture{annotation}{greannotation}{string}

%% A-bove L-ines T-ext shall be italicized and in a smaller font
\grechangestyle{abovelinestext}{\itshape\fontsize{8}{10}\selectfont}
%% for psalm tones etc. Roman/upright text is what the Liber Usualis uses. 
\NewDocumentCommand{\altnormal}{}{\grechangestyle{abovelinestext}{\normalfont\fontsize{8}{10}\selectfont}}
%%  resets ALT font
\NewDocumentCommand{\altitshape}{}{\grechangestyle{abovelinestext}{\itshape\fontsize{8}{10}\selectfont}} 

\NewDocumentCommand{\altlarge}{}{\grechangestyle{abovelinestext}{\large}} %% for I reading of Lamentations

\NewDocumentCommand{\altupright}{}{\grechangestyle{abovelinestext}{\normalfont}} %%


%% fine-tuning of space beween the staff and the text above lines (used for Magnificats; needs to be rethought (again, should 2 and 3 be raised or not?) and worked into \smallscore
\newcommand{\altraise}{0.1cm} %% default is -0.1cm %% needs to be fine-tuned for \rubrique command with EB Garamond font. -0.2cm is MB value
\grechangedim{abovelinestextraise}{\altraise}{scalable}

\newcommand{\altheight}{0.5cm} %% default is 0.3cm and value must be bigger than text height; this should fix the problem. but needs further investigation.
\grechangedim{abovelinestextheight}{\altheight}{scalable} %% this is a very finicky command.

\newcommand{\comraise}{0.4cm} %% default is 0,2m %%
\grechangedim{commentaryraise}{\comraise}{scalable}

%% allows printing of glyphs from Gregorio score font (available glyphs listed in documentation) as text.
\makeatletter
\def\gretextglyph#1{{\gre@font@music\csname GreCP#1\endcsname}}
\makeatother

%%% Font %%%
\setmainfont{EB Garamond}[UprightFont=EB Garamond Regular,
ItalicFont= EB Garamond Italic,
BoldFont= EB Garamond Bold,
Ligatures=Rare,
Numbers=Proportional,
Numbers=OldStyle] %% all \kern numbers are based on this and can be removed if this font is abandoned.
\newfontfamily\symbolfont{liturgy}
  %% text must be written {\symbolfont{+}} (you can use V and R as well) but may conflict as + is † in gabc. Cross should be \small or even \footnotesize


%    \grechangestyle{initial}{\fontspec{EB Garamond Initials}\fontsize{#1}{#1}\selectfont}

%%% default initial size is 32 points
%\newcommand{\defaultinitialsize}{28}
%\initialsize{\defaultinitialsize}

%%%% Font %%%
%\setmainfont{EB Garamond}[UprightFont=EB Garamond Regular,
%ItalicFont= EB Garamond Italic,
%BoldFont= EB Garamond Bold,
%Ligatures=Rare,
%StylisticSet=6,
%Numbers=Proportional,
%Numbers=OldStyle] %% all \kern numbers are based on this and can be removed if this font is abandoned. %%StylisticSet=6 is, in Pardo EBGaramond, the long-tailed Q.
%\newfontfamily\symbolfont{liturgy} 
  %% text must be written {\symbolfont{+}} (you can use V and R as well) but may conflict as + is † in gabc. Cross should be \small or even \footnotesize


%%% TYPOGRAPHICAL AND SYMBOL COMMANDS  %%%%

%% \P is pilcrow and is defined in LaTeX as is.
%% Add {\textcolor{gregoriocolor} as first part of command's definition if changing to red.


%%for proper spacing of all-caps letters to prevent clashes, e.g. serif strokes touching.

\NewDocumentCommand\capspace{m}{{\addfontfeature{LetterSpace=5.0}{#1}}}
\NewDocumentCommand\scspace{m}{\textsc{{{\addfontfeature{LetterSpace=5.0}{#1}}}}}

\NewDocumentCommand\feasttitle{m}{{\centering{\capspace{\huge{#1}}}\par}}


%% macro to print Alleluia for versicles.
\NewDocumentCommand{\tpalleluia}{}{(\textit{T.P.} Allelúia.)}

\newcommand{\specialcharhsep}{3mm} % space after invoking R/ or V/ or A/ outside rubrics
%\newcommand{\vv}{\textcolor{gregoriocolor}{\fontspec[Scale=\customscale]{Charis SIL}℣.\nolinebreak[4]\hspace{\specialcharhsep}\nolinebreak[4]}} %% format from MB

\newcommand{\vv}{{\normalfont ℣.\nolinebreak[4]\hspace{\specialcharhsep}\nolinebreak[4]}}
\newcommand{\rr}{{\normalfont ℟.\nolinebreak[4]\hspace{\specialcharhsep}\nolinebreak[4]}}

%% typesets a cross pattée.
\NewDocumentCommand{\cc}{}{%
    % Grouping to keep font changes local
    {%
        % Ensure we're not in italics (since liturgy.ttf doesn't have italic)
        \normalfont%
        % Select the font liturgy.ttf
        \symbolfont%
        % Set the size
        \footnotesize%
        % In liturgy.ttf, the plus (+) is a cross pattée
        +%
    }%    
} %%  can replace <+> with <U+2720> (see above) if a suitable font is found (or EB Garamond font is fixed); command will be useful in lieu of typing the Unicode.
%% use <v> tag to insert into a gabc score (usually for the faithful or for blessings, e.g. the font.

%% Same special characters, for in-score use (<sp>V/ R/ A/ +</sp>)

\gresetspecial{V/}{{\fontspec[Scale=\customscale]{EB Garamond}℣.~}}
\gresetspecial{R/}{{\fontspec[Scale=\customscale]{EB Garamond}℟.~}}
\gresetspecial{+}{{\fontspec[Scale=\customscale]{EB Garamond}†~}}
\gresetspecial{*}{\gresixstar}
\gresetspecial{cross}{\cc}

%% Same special characters, for use in rubrics (no space)

\newcommand{\vvrub}{{\normalfont ℣.~}}
\newcommand{\rrrub}{{\normalfont ℟.~}}

%%rubrics: black italics, smaller than body of psalms etc

\NewDocumentCommand{\rubrique}{m}{%
    {%
        \fontsize{10}{12}%
        \selectfont%
        \textit{%
        %
        #1%
    }%
    }}
    
%% macro to print normal text inside of rubric (name of a chant or prayer, etc.)

\NewDocumentCommand{\normaltext}{m}{%
    {%
        \normalfont%
        #1%
    }%
    }
    
%% in case something should be bolded inside of a rubrique
\NewDocumentCommand{\rubriquegras}{m}{%
    {%
        \normalfont%
\textbf{#1%
}%
}}

%% to print in red instead of italicizing
\NewDocumentCommand{\rouge}{m}{%
\textcolor{gregoriocolor}%
{\fontsize{10}{12}%
\selectfont{%
#1%
}%
}}

%% to print in black within rouge group.

%\NewDocumentCommand{\textenoir}{m}{%

\NewDocumentCommand{\textenoir}{m}{%
\textcolor{black}%
{%
\normalfont%
#1%
}%
}

%% rouge but in italic (this is technically not correct).
\NewDocumentCommand{\rougeit}{m}{%
\textcolor{gregoriocolor}%
{\fontsize{10}{12}%
\selectfont%
\textit{#1%
}%
}}

%%for Scriptural references of the chapter.
\NewDocumentCommand{\scripture}{m}{%
{\raggedleft{\textit{#1}%
}%
\par}%
}

%\newcommand{\rubriquegras}[1]{{\fontsize{9}{11}\selectfont\textbf{#1}}}

%%macro to space punctuation with ecclesiasticlatin language from babel.
\NewDocumentCommand{\espaceponctuation}{}{\hspace{0.10em}} %%this value seems more balanced with this font than the definition provided in the ecclesiasticlatin documentation.

\NewDocumentCommand{\myrule}{}{%
    \par%
    {%
        \centering%
        \rule{0.3\textwidth}{0.4pt}%
        \par%
    }% typesets a horizontal rule like on the page with the prayers before and after the office.
}  %%If you're using color at all in your document, you might want to either force \myrule to use black or make it cusotmizable. (from u/Independent-Comb-257)

\NewDocumentCommand{\bigrule}{}{%
    \par%
    {%
        \centering%
        \rule{0.6\textwidth}{0.4pt}%
        \par%
    }% typesets a horizontal rule like on the page with the prayers before and after the office.
}  %%If you're using color at all in your document, you might want to either force \myrule to use black or make it cusotmizable. (from u/Independent-Comb-257)


%%% typesets a cross pattée.

\NewDocumentCommand{\mycross}{}{%
{\symbolfont\small{{+}}}%
{}
} %% can replace <+> with <U+2720> if a suitable font is found (or EB Garamond font is fixed); command will be useful in lieu of typing the Unicode.

%% macro to format psalm text

%\NewDocumentCommand{\pstexte}{ m }{%
%    \smallskip%
%    \noindent%
%    \begin{itemize}[%
%    		label=\null, %
%			leftmargin=0pt, %
%			itemindent=0mm, %
%			labelsep=0pt, %
%			labelwidth=0pt, %
%			rightmargin=0pt, %
%			parsep=0pt, %
%			topsep=0pt, %
%			itemsep=0pt]%
%    \input{psaumes/#1}
%    \end{itemize}}
          
      \setlength{\columnsep}{3pc}
%\setlength{\columnsep}{10pt}
\setlength\columnseprule{0.4pt}


%% original version kept so that I can fix errors found later and keep using common texts, going forward
    \NewDocumentCommand{\textebilingue}{ mm }{%
    \smallskip%
       \begin{paracol}{2}
           \input{psaumes/#1}
           \switchcolumn
    \input{psaumes/#2}
    \end{paracol}}

%%new version;  no need for input (too many chapter/collect files) + is the equivalent of long, allows \par
    \NewDocumentCommand{\texteparacol}{ +m+m }{%
    \smallskip%
       \begin{paracol}{2}
           #1
           \switchcolumn
    #2
    \end{paracol}}
    
    \NewDocumentCommand{\pstexte}{mm}{%
    \smallskip%
    \noindent%%
   \begin{paracol}{2}
    \begin{enumerate}[%
    		label=\arabic*., %
			leftmargin=3pt, %
			itemindent=3mm, %
			labelsep=3pt, %
			labelwidth=0pt, %
			rightmargin=0pt, %
			parsep=0pt, %
			topsep=0pt, %
			itemsep=0pt,%
			start=2]%
    \input{psaumes/#1.tex}
    \end{enumerate}
    \switchcolumn
       \begin{enumerate}[%
    		label=\arabic*.,%
			leftmargin=3pt, %
			itemindent=3mm, %
			labelsep=3pt, %
			labelwidth=0pt, %
			rightmargin=0pt, %
			parsep=0pt, %
			topsep=0pt, %
			itemsep=0pt]%
    \input{psaumes/#2.tex}
      \end{enumerate}
    \end{paracol}%
    } %% modified from original in order to allow 2 column psalms
    
    %% Command de Matthias Bry modifié, prints psalm incipit score with the text
    \NewDocumentCommand{\psalmus}{mmmm}{%
	\needspace{4\baselineskip}%
	\smalltitle{Psalmus #1.}%
	\smallscore[n]{#2}%
	\pstexte{#3}{#4}%
}

    %% sd=semiduplex. especially for Dominica ad Vesperas in the Psalterium (the only example which comes to mind, in fact!)
 \NewDocumentCommand{\sdpsalmusbilingue}{mmm}{%
	\needspace{4\baselineskip}%
		\smalltitle{Psalmus #1.}%
	\pstexte{#2}{#3}%
	}

    %% Command de Matthias Bry modifié, prints canticle incipit score with the text and scriptural commentary
    \NewDocumentCommand{\canticum}{mmmm}{
	\needspace{4\baselineskip}
	\smalltitle{Canticum #1.}
	 \grecommentary{#2}
	\smallscore[n]{#3}
	\pstexte{#4}
}
	
    %% macro to print any additional text (Capitiulum, oratio, rubrics)
 \NewDocumentCommand{\textes}{ m }{%
    \input{textes/#1}}
    
%     \NewDocumentCommand{\lection}{m}{%
%   \begin{multicols}{2}#1\end{multicols}} %% this doesn't work as expected, needs to be fixed
    
    
    
    %% SC work for page headers and for secondary headings but not so much for things like "Dominica…" and certainly not the feast name%%
    
    %%%% Headers%%%%
    \pagestyle{fancy}
\fancyhead{}
\fancyfoot{}
\renewcommand{\headrulewidth}{0pt}



\NewDocumentCommand{\zerobaroffsettextleft}{}%
{%
\grechangedim{maxbaroffsettextleft}%
{0cm}%
{scalable}%
}

\NewDocumentCommand{\zerobaroffsettextright}{}%
{%
\grechangedim{maxbaroffsettextright}%
{0cm}%
{scalable}%
}

%%%the default for left bar offset is 0.3cm. Right is 0.15cm. This needs to be applied after a score where the bar offset is modified.

\NewDocumentCommand{\resetbaroffsettextleft}{}%
{%
\grechangedim{maxbaroffsettextleft}%
{0.3cm}%
{scalable}%
}

\NewDocumentCommand{\resetbaroffsettextright}{}%
{%
\grechangedim{maxbaroffsettextright}%
{0.15cm}%
{scalable}%
}



 \setlength{\headheight}{13.59999pt} %% recommended by LaTeX

%\fancyhead[RO]{\small\thepage}
%\fancyhead[LE]{\small\thepage}

\fancyhead[RO]{\small\rightmark\hspace{0.5cm}\thepage}
\fancyhead[LE]{\small\thepage\hspace{0.5cm}\leftmark}

\newcommand{\setheaders}[2]{
	\renewcommand{\rightmark}{{\sc#2}}
	\renewcommand{\leftmark}{{\sc#1}}
}
\setheaders{}{}


%% TITLE Commands %%%% %% TITLE Commands %%%% currently dead but Matthias says this isn't a good idea.
\titleformat{\section}[display]{\huge\filcenter\addfontfeature{LetterSpace=5.0}}{}{}{}
\titleformat{\subsection}[display]{\large\scshape\filcenter\addfontfeature{LetterSpace=5.0}}{}{}{}
\setcounter{secnumdepth}{0}
\titlespacing{\section}{0pt}{*0}{*0}


% \NewDocumentCommand{\mysection}{m}{\section{#1}\setheaders{{\scshape\addfontfeature{LetterSpace=5.0}#1}}{{\scshape\addfontfeature{LetterSpace=5.0} #1}}}
 
\NewDocumentCommand{\header}{m}{\setheaders{{\scshape\addfontfeature{LetterSpace=5.0}#1}}{{\scshape\addfontfeature{LetterSpace=5.0} #1}}}

%% all of these are a mess and should probably be ignored, but feel free to use them by referring to Matthias B's originals in the Nocturnale Romanum repository.

\newcommand{\officiumtitulum}[1]{
%  \newpage
%  \thispagestyle{empty}
  \setheaders{{\scshape\addfontfeature{LetterSpace=3.0}#1}}{{\scshape\addfontfeature{LetterSpace=3.0} #1}}
  \begin{center}
  {\scshape\large #1}\par
  \end{center}}
  
%   \setheaders{{{\scshape\addfontfeature{LetterSpace=3.0}}}{{{\scshape\addfontfeature{LetterSpace=3.0} {#1}}}

\NewDocumentCommand{\rank}{m}{
\needspace{5\baselineskip}  %% very small if there are neumes above the staff, including flats in mode 2, e.g. O Doctor optime
\vspace{\baselineskip}
 {\centering{\textit{\large #1}}\par}
}

\NewDocumentCommand{\bigtitle}{m}{
\needspace{5\baselineskip}  %% very small if there are neumes above the staff, including flats in mode 2, e.g. O Doctor optime
\vspace{\baselineskip}
 {\centering{\large#1}\par}
}

\NewDocumentCommand{\smalltitle}{m}{
\needspace{5\baselineskip}  %% very small if there are neumes above the staff, including flats in mode 2, e.g. O Doctor optime
\vspace{\baselineskip}
 {\centering #1\par}
}

\NewDocumentCommand{\chapterlateng}{}{%
\smalltitle{Chapter.}%
}


%\newcommand{\subtitulum}[1]{
% \subsection{
%  \begin{center}
%  {\large#1}\par
%  \end{center}}}
  
  %% something has broken here
  %% see https://tex.stackexchange.com/questions/545841/paragraph-ended-before-ttlformatsi-was-complete

%% originally used apparently for typesetting portions of the Common of the BVM
\newcommand{\espacetitre}{\vspace{-0.5cm}}
\newcommand{\espaceps}{\vspace{-3mm}} %% ps =psaume
\newcommand{\espacecap}{\vspace{-3mm}} %% cap is capitulum
\newcommand{\espace}{\vspace{2mm}}

\NewDocumentCommand{\secreto}{}{Pater noster. \rubrique{secreto.}}

\NewDocumentCommand{\pateravecredo}{}{\large{Pater noster, Ave María, \textit{et} Credo.}}

\NewDocumentCommand{\prayers}{}{%
Our Father and Hail Mary, \rubrique{silently.}%
}

%%for litanies
\makeatletter  
\newcounter{score}
\newcounter{tabstop}[score]
\newcommand{\grealign}{%
	\@bsphack%
	\ifgre@boxing\else%
		\kern\gre@dimen@begindifference%
		\stepcounter{tabstop}%
		\expandafter\zsavepos{stop-\thescore-\thetabstop}%
		\kern-\gre@dimen@begindifference%
	\fi%
	\@esphack%
}

\newcommand{\setstops}{%
  \gdef\nstabbing@stops{%
    \hspace*{-\oddsidemargin}\hspace{-1in}%
    \hspace*{\zposx{stop-\thescore-1} sp}\=%
  }%
  \count@=\@ne
  \loop\ifnum\count@<\value{tabstop}%
    \begingroup\edef\x{\endgroup
      \noexpand\g@addto@macro\noexpand\nstabbing@stops{%
        \noexpand\hspace{-\noexpand\zposx{stop-\thescore-\the\count@} sp}%
        \noexpand\hspace{\noexpand\zposx{stop-\thescore-\the\numexpr\count@+1} sp}\noexpand\=%
      }%
    }\x
    \advance\count@\@ne
  \repeat
  \nstabbing@stops\kill
}
\makeatother

\newenvironment{nstabbing}
  {\setlength{\topsep}{0pt}%
   \setlength{\partopsep}{0pt}%
   \tabbing%
   \setstops}
  {\endtabbing\stepcounter{score}}

\makeatletter
\newcounter{blankpages} %% should remove number on blank page at end of preface. Cf https://tex.stackexchange.com/questions/250761/how-to-not-count-blank-pages-in-frontmatter-of-two-sided-document
\def\cleardoublepage{%
    \clearpage
    \if@twoside
    \ifodd\c@page
    \else
    \hbox{}
    \thispagestyle{empty}%
    \newpage\stepcounter{blankpages}%
    \if@twocolumn\hbox{}\newpage\fi
    \fi
    \fi
}
\newcommand{\@romannoblank}[1]{%
    \@roman{\numexpr#1-\value{blankpages}\relax}%
}
\makeatother

\providecommand\phantomsection{} %% should make hyperref work properly %%only works with section etc., not with label etc. (that requires \phantomsection
%
\renewcommand{\contentsname}{\hfill\bfseries\LARGE Index Generalis.\hfill\null}  %% modifies TOC title
\renewcommand{\cfttoctitlefont}{\hfil\LARGE}

\begin{document}

\thispagestyle{empty}

\bigtitle{Benediction on Sundays of June.}

\rubrique{The cantors intone until the asterisk or the quarter bar.}

\rubrique{Alternate medieval melody restored in the Vatican Edition.}
\grechangestyle{initial}{\fontsize{28}{28}\selectfont}
\greannotation{7.}

\gabcsnippet{(c3) O(e) sa(fvED')lu(e)tá(f')ris(e) Hó(f!gw!h!g)sti(fe)a,(e.) (,) Quæ(g) cæ(iji')li(h) pan(gf)dis(e) ó(f')sti(d)um,(e.) (;) Bel(g)la(i') pre(i)munt(j') ho(i)stí(h')li(g)a,(fge.) (,) Da(e) ro(fvED')bur,(e) fer(f') au(e)xí(f!gw!h!g)li(fe)um.(e.) (::) 
2. U(e)ni(fvED') tri(e)nó(f')que(e) Dó(f!gw!h!g)mi(fe)no(e.) (,) Sit(g) sem(iji')pi(h)tér(gf)na(e) gló(f')ri(d)a,(e.) (;) Qui(g) vi(i')tam(i) si(j')ne(i) tér(h')mi(g)no(fge.) (,) No(e)bis(fvED') do(e)net(f') in(e) pá(f!gw!h!g)tri(fe)a.(e.) (::) 
A(efe)men.(de..) (::)}

\begin{multicols}{2}
\raggedcolumns
\begin{otherlanguage}{english}
O saving victim, opening wide\\
The gate of heaven to man below,\\
Our foes press on from every side,\\
Thine aid supply, thy strength bestow.

To thy great name be endless praise,\\
Immortal Godhead, One in Three;\\
O grant us endless length of days\\
In our true native land, with thee.
Amen.
\end{otherlanguage}
\end{multicols}

\rubrique{The Litany of the Sacred Heart is said.}

\begin{otherlanguage}{english}
{\centering{Prayer Before Benediction on Sundays.\kern 0.05em\footnote{\raggedright{Prayer adapted from Nicholas Stephen Cardinal Wiseman, Archbishop of\kern 0.25em West\-min\-ster (1802–1865).}}}\par}

\rubrique{This is said on all Sundays except for the second of each month.}

\rubrique{All kneel; the Priest begins:}

\lettrine{O}{} Blessed Virgin Mary, (\textit{all join}) Mother of God | and our most gentle Queen and Mother, | look down in mercy upon us all | who greatly hope and trust in thee. | By thee it was that Jesus | our Savior and our Hope | was given unto the world; | and He has given thee to us | that we might hope still more.

Plead for us, thy children, | whom thou didst receive and accept | at the foot of the Cross, | O sorrowful Mother. | Intercede for our separated brethren, | that with us, | in the one true fold, | they may be united to the supreme Shepherd, | the Vicar of thy Son.

Pray for us all, dear Mother | that by faith | fruitful in good works, | we may all deserve to see and praise God, | together with thee, | in our heavenly home. Amen.

\rubrique{On the Second Sunday of each month, instead of the prayer \normaltext{O Blessed Virgin Mary,} the following prayer is said, all kneeling.}

\rubrique{The priest says:}

\lettrine{O}{} merciful God, let the glorious intercession of Thy saints assist us; above all, the most blessèd Virgin Mary, the Immaculate Mother of Thine only-begotten Son, and Thy holy apostles, Peter and Paul, through whose patronage we humbly commend this land. Be mindful of our fathers, Thy glorious bishop John Neumann; of Junípero and Damian, Thy priests. Remember our holy martyrs who shed their blood for Christ: Isaac, René and Jean. Remember those holy virgins and widows: Frances, Rose-Philippine, Katherine, Théodore, Marianne, Kateri, and Elizabeth Ann; and all those holy men and women who made this country illustrious by their glorious merits and virtues. Let not thy memory perish before Thee, O Lord, but let their supplication enter daily into Thy sight; and do Thou, who didst so often spare Thy sinful people for the sake of Abraham, Isaac, and Jacob, now, also moved by the prayers of our fathers, brothers, and sisters reigning with Thee, have mercy up on us, save Thy people and bless Thine inheritance; and suffer not those souls to perish which Thy Son hath redeemed with His own most Precious Blood: Who liveth and reignest with Thee, world without end. ℟. Amen.
%\smallskip

Let us pray.

\lettrine{O}{} most loving Lord Jesus, Who, when Thou wert hanging on the Cross, didst commend us all in the person of Thy disciple John to Thy most sweet Mother, that we might find in her our refuge, our solace and our hope; look graciously upon our beloved land, and on those who are bereaved of so powerful a patronage; that acknowledging the dignity of this Holy Virgin, they may honor and venerate her with all affection of devotion, and own her as Queen and Mother. May her sweet name be listed by little ones, and linger on the lips of the agèd and dying; and may it be invoked by the afflicted, and hymned by the joyful; that this Star of the Sea being their protection and their guide, all may come to the harbor of eternal salvation. Who liveth and reignest with Thee, world without end. ℟. Amen.

\rubrique{Then follows Benediction of the Most Blessed Sacrament of the Eucharist, in the usual way.}

\rubrique{Modern chant, based on the familiar tune attributed to John Francis Wade, ca. 1751.}

\end{otherlanguage}
\grechangestyle{initial}{\fontsize{28}{28}\selectfont}
\greannotation{5.}
\gabcsnippet{(c4) TAn(c)tum(d) er(de)go(dc) Sa(d)cra(eg)mén(gf)tum(e.) (,)
Ve(h')ne(g)ré(gf)mur(e) cér(ed)nu(c)i:(c.) (;)
Et(j) an(i')tí(j)quum(g.) do(h)cu(g)mén(gf)tum(e.) (,)
No(h)vo(i) ce(j')dat(i) rí(ih)tu(g)i:(g.) (:)
Præ(g)stet(g) fi(e)des(dc) sup(d)ple(eg)mén(gf)tum(e.) (,)
Sén(h)su(g)um(gf) de(e)fé(ed)ctu(c)i.(c.) (::) (Z-) 
2. Ge(c)ni(d)tó(de)ri,(dc) Ge(d)ni(eg)tó(gf)que(e.) (,)
Laus(h) et(g) ju(gf)bi(e)lá(ed)ti(c)o,(c.) (;)
Sa(j)lus,(i') ho(j)nor,(g.) vir(h)tus(g) quo(gf)que(e.) (,)
Sit(h) et(i) be(j')ne(i)dí(ih)cti(g)o:(g.) (:)
Pro(g)ce(g')dén(e)ti(dc) ab(d) u(eg)tró(gf)que(e.) (,)
Com(h)par(g) sit(gf) lau(e)dá(ed)ti(c)o.(c.) (::) A(cdc)men.(bc..) (::)}
%\newpage
\begin{multicols}{2}
\raggedcolumns
\begin{otherlanguage}{english}
Down in adoration falling,\\
Lo! the sacred Host we hail;\\
Lo! o'er ancient forms departing,\\
Newer rites of grace prevail;\\
Faith for all defects supplying,\\
Where the feeble senses fail.

To the everlasting Father,\\
And the Son who reigns on high,\\
With the Holy Ghost proceeding\\
Forth from each eternally,\\
Be salvation, honour, blessing,\\
Might, and endless majesty.
Amen.
\end{otherlanguage}
\end{multicols}

   \begin{paracol}{2}
\sloppy
%\raggedcolumns
\vspace{-0.5\baselineskip}

\noindent \vv Panem de cælo præstitisti eis. \tpalleluia{}\footnote{Atque per octavam Corporis Christi. \textit{As well as throughout the octave of Corpus Christi.}}

\noindent \rr Omne delectamentum in se habentem. \tpalleluia{}

  Orémus.
  
  \lettrine{D}{e}us, qui nobis sub Sacraménto mirábili passiónis tuæ memóriam reliquísti:~† tríbue, quǽsumus, ita nos córporis, et sánguinis tui sacra mystéria venerári;~* ut redemptiónis tuæ fructum in nobis júgiter sentiámus. Qui vivis et regnas in sǽcula sæculórum. 

  \switchcolumn \sloppy
\begin{otherlanguage}{english}
\vspace{-0.5\baselineskip}
\noindent \vv Thou hast given them bread from heaven. (\textit{P.T.} alleluia.)

\noindent \rr Having all sweetness within it. (\textit{P.T.} alleluia.)

\noindent Let us pray. O God, who under a wonderful Sacrament hast left us a memorial of Thy Passion; grant us, we beseech Thee, so to reverence the sacred mysteries of Thy Body and Blood, that we may ever feel within ourselves the fruit of Thy Redemption. Who livest and reignest forever and ever. Amen.
\end{otherlanguage}\end{paracol}

\begin{multicols}{2}
\raggedcolumns
\begin{otherlanguage}{english}

Blessed be God. 

Blessed be His Holy Name. 

Blessed be Jesus Christ, true God and true Man.
 
Blessed be the Name of Jesus.

Blessed be His Most Sacred Heart.

Blessed be His Most Precious Blood.

Blessed be Jesus in the Most Holy Sacrament of the Altar.

Blessed be the Holy Spirit, the Paraclete.

Blessed be the great Mother of God, Mary most Holy.

Blessed be her Holy and Immaculate Conception.

\rubrique{The following invocation is said thrice.}

Blessed be her Glorious Assumption.

Blessed be the name of Mary, Virgin and Mother.

Blessed be Saint Joseph, her most chaste spouse.

Blessed be God in His Angels and in His Saints.

\end{otherlanguage}
\end{multicols}

%\rubrique{These are sung alternating between a cantor and the whole congregation, which also sings \normaltext{Fiat, Fiat} at the end.}

%\gscore[]{}{laudes_divinae_no_breaks}

\begin{otherlanguage}{english}\rubrique{The cantors intone until the half bar, then the people are welcome to sing the rest of the antiphon and the repetitions.}\end{otherlanguage}

\rubrique{Melody adapted, probably by Dom Joseph Pothier, o.s.b.}

\gscore[]{1.}{va--cor_jesu--solesmes}
\begin{otherlanguage}{english}Most Sacred Heart of Jesus, have mercy on us.
\end{otherlanguage}


\enddocument