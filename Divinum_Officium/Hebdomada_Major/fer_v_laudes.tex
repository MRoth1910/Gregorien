% !TEX TS-program = LuaLaTeX+se
% !TEX root = Officium_Tenebrarum.tex

\subsection[Ad Laudes.]{ad laudes.}

\gscore[1. Ant.]{8. G}{an_justificeris_domine}
\psalmus{50}{50_8G}{50_8G}

\gscore[2. Ant.]{2. D}{an_dominus_tamquam_ovis}
\psalmus{89}{89_2}{89_2}

\gscore[3. Ant.]{8. G}{an_contritum_est_cor_meum}
\psalmus{35}{35_8G}{35_8G}

\gscore[4. Ant.]{4. A*}{an_exhortatus_es}
\canticum{Moysi}{Exodi 15, 1–19.}{Canticum_Moysis_Exod_4_alt_Astar}{Canticum_Moysis_Exod_4_A}

\gscore[5. Ant.]{2. D}{an_oblatus_est}
\psalmus{146}{146_2}{146_2}

\textes{rubrique_capitule}

\smallscore{vr_homo_pacis}

\smalltitle{Ad Benedictus, Antiphona.}

\gscore[]{1. g}{an_traditor_autem}

\phantomsection\label{Benedictus}
\canticum{Zachariæ}{Luc. 1, 68–79}{Benedictus_1g_solemn}{Benedictus_1g_solemn}

\textes{rubrique_Benedictus}

\phantomsection\label{Christus factus est}
\initialscore[5.]{an_christus_factus_est}
\textes{rubrique_secunda_nocte}
\smallscore{an_mortem_autem}
\textes{rubrique_tertia_nocte}
\smallscore{an_propter_quod}

\textes{rubrique_miserere}

\textes{Miserere}

\textes{Oratio_ad_Horas}
\newpage