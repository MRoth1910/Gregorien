\bigtitle{Lectio V.}

\lettrine{N}{o}stis qui convéntus erat malignántium Judæórum, et quæ multitúdo erat operántium iniquitátem. Quam iniquitátem? Quia voluérunt occídere Dóminum Jesum Christum. Tanta ópera bona, inquit, osténdi vobis: propter quod horum me vultis occídere? Pértulit omnes infírmos eórum, curávit omnes lánguidos eórum, prædicávit regnum cælórum, non tácuit vítia eórum, ut ipsa pótius eis displicérent, non médicus, a quo sanabántur. His ómnibus curatiónibus ejus ingráti, tamquam multa febre phrenétici, insaniéntes in médicum, qui vénerat curáre eos, excogitavérunt consílium perdéndi eum: tamquam ibi voléntes probáre, utrum vere homo sit, qui mori possit, an áliquid super hómines sit, et mori se non permíttat. Verbum ipsórum agnóscimus in Sapiéntia Salomónis: Morte turpíssima, ínquiunt, condemnémus eum. Interrogémus eum: erit enim respéctus in sermónibus illíus. Si enim vere Fílius Dei est, líberet eum.