\bigtitle{Lectio VI.}

\lettrine{E}{x}acuérunt tamquam gládium linguas suas. Non dicant Judǽi: Non occídimus Christum. Etenim proptérea eum dedérunt júdici Piláto, ut quasi ipsi a morte ejus videréntur immúnes. Nam cum dixísset eis Pilátus: Vos eum occídite: respondérunt, Nobis non licet occídere quemquam. Iniquitátem facínoris sui in júdicem hóminem refúndere volébant: sed numquid Deum júdicem fallébant? Quod fecit Pilátus, in eo ipso quod fecit, aliquántum párticeps fuit: sed in comparatióne illórum multo ipse innocéntior. Institit enim quantum pótuit, ut illum ex eórum mánibus liberáret: nam proptérea flagellátum prodúxit ad eos. Non persequéndo Dóminum flagellávit, sed eórum furóri satisfácere volens: ut vel sic jam mitéscerent, et desínerent velle occídere, cum flagellátum vidérent. Fecit et hoc. At ubi perseveravérunt, nostis illum lavísse manus, et dixísse, quod ipse non fecísset, mundum se esse a morte illíus. Fecit tamen. Sed si reus, quia fecit vel invítus: illi innocéntes, qui coëgérunt ut fáceret? Nullo modo. Sed ille dixit in eum senténtiam, et jussit eum crucifígi, et quasi ipse occídit: et vos, o Judǽi, occidístis. Unde occidístis? Gládio linguæ: acuístis enim linguas vestras. Et quando percussístis, nisi quando clamástis: Crucifíge, crucifíge?