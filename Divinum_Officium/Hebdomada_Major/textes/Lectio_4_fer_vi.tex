\bigtitle{Lectio IV.\\ Ex tractátu sancti Augustíni Epíscopi super Psalmos.}

{\hfill \textit{In Psalm 63 ad versum 2.}}

\lettrine{P}{r}otexísti me, Deus, a convéntu malignántium, a multitúdine operántium iniquitátem. Jam ipsum caput nostrum intueámur. Multi Mártyres tália passi sunt, sed nihil sic elúcet, quómodo caput Mártyrum: ibi mélius intuémur, quod illi expérti sunt. Protéctus est a multitúdine malignántium, protegénte se Deo, protegénte carnem suam ipso Fílio, et hómine, quem gerébat: quia fílius hóminis est, et Fílius Dei est. Fílius Dei, propter formam Dei: fílius hóminis, propter formam servi, habens in potestáte pónere ánimam suam, et recípere eam. Quid ei potuérunt fácere inimíci? Occidérunt corpus, ánimam non occidérunt. Inténdite. Parum ergo erat, Dóminum hortári Mártyres verbo, nisi firmáret exémplo.