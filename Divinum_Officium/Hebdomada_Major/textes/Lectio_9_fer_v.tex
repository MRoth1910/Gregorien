\bigtitle{Lectio IX.}

\lettrine{I}{t}aque quicúmque manducáverit panem hunc, vel bíberit cálicem Dómini indígne, reus erit córporis et sánguinis Dómini. 
Probet autem seípsum homo: et sic de pane illo edat, et de cálice bibat. 
Qui enim mandúcat et bibit indígne, judícium sibi mandúcat et bibit, non dijúdicans corpus Dómini. 
Ideo inter vos multi infírmi et imbecílles, et dórmiunt multi. 
Quod, si nosmetípsos dijudicarémus, non útique judicarémur. 
Dum judicámur autem, a Dómino corrípimur, ut non cum hoc mundo damnémur. 
Itaque, fratres mei, cum convenítis ad manducándum, ínvicem exspectáte. 
Si quis ésurit, domi mandúcet: ut non in judícium conveniátis. Cétera autem, cum vénero, dispónam.