\bigtitle{Lectio IV.\\ Ex tractátu sancti Augustíni Epíscopi super Psalmos.}

{\hfill \textit{In Psalmum 54 ad 1 versum.}}

\lettrine{E}{x}áudi, Deus, oratiónem meam, et ne despéxeris deprecatiónem meam: inténde mihi, et exáudi me. Satagéntis, sollíciti, in tribulatióne pósiti, verba sunt ista. Orat multa pátiens, de malo liberári desíderans. Súperest ut videámus in quo malo sit: et cum dícere cœ́perit, agnoscámus ibi nos esse: ut communicáta tribulatióne, conjungámus oratiónem. Contristátus sum, inquit, in exercitatióne mea, et conturbátus sum. Ubi contristátus? ubi conturbátus? In exercitatióne mea, inquit. Hómines malos, quos pátitur, commemorátus est: eandémque passiónem malórum hóminum exercitatiónem suam dixit. Ne putétis gratis esse malos in hoc mundo, et nihil boni de illis ágere Deum. Omnis malus aut ídeo vivit, ut corrigátur; aut ídeo vivit, ut per illum bonus exerceátur.