\documentclass[11pt]{book}
\usepackage[paperheight=9in,paperwidth=6in]{geometry}
%\pagenumbering{gobble}
\usepackage{fontspec}
\usepackage{fancyhdr}
\usepackage[compact]{titlesec}
\usepackage{needspace}

\usepackage[autocompile]{gregoriotex}
\grechangestyle{abovelinestext}{\it\fontsize{9}{11}\selectfont}
\def\GreStar{*}
\def\GreDagger{†}

\newcommand{\smallscore}[2][y]{
  \gresetinitiallines{0}
  \gregorioscore{psaumes/#2}
  \gresetinitiallines{1}
}

\newcommand{\vscore}[2][y]{
  \gresetinitiallines{0}
  \gregorioscore{Versus_in_Festo/#2}
  \gresetinitiallines{1}
}
%%% Font %%%
\setmainfont{EB Garamond}[UprightFont=EB Garamond Regular,
ItalicFont= EB Garamond Italic,
BoldFont= EB Garamond Bold,
Ligatures=Rare,
StylisticSet=6,
Numbers=Proportional,
Numbers=OldStyle]

\newcommand{\rubrique}[1]{{\fontsize{9}{11}\selectfont\textit{#1}}}
\newcommand{\normaltext}[1]{{\normalfont\fontsize{9}{11}\selectfont{#1}}}

\newcommand{\smalltitle}[1]{
\needspace{5\baselineskip}
\vspace{\baselineskip}
 {\centering #1\par}
}
%\renewenvironment{center}[1]{%
%  \begin{center}
%  {#1} 
%\vspace{-3mm}}{%
%  \end{center}
%}
%% Input Commands %%%
\NewDocumentCommand{\pstext}{ m }{%
    \smallskip%
    \noindent%
    \input{Psaumes/#1}}
    
    \NewDocumentCommand{\Capitulum}{ m }{%
    \input{Capitulum/#1}}
    %%%% Headers%%%%
    \pagestyle{fancy}
\fancyhead{}
\fancyfoot{}
\renewcommand{\headrulewidth}{0pt}

%\fancyhead[RO]{\small\thepage}
%\fancyhead[LE]{\small\thepage}

\fancyhead[RO]{\small\rightmark\hspace{1cm}\thepage}
\fancyhead[LE]{\small\thepage\hspace{1cm}\leftmark}

\newcommand{\setheaders}[2]{
	\renewcommand{\rightmark}{{\sc#2}}
	\renewcommand{\leftmark}{{\sc#1}}
}
\setheaders{}{}


%% TITLE Commands %%%%
\titleformat{\section}[display]{\large\filcenter\sc\addfontfeature{LetterSpace=3.0}}{}{}{}
\setcounter{secnumdepth}{0}
\titlespacing{\section}{0pt}{*0}{*0}

\titleformat{\subsection}[display]{\large\filcenter\sc\addfontfeature{LetterSpace=3.0}}{}{}{}

\newcommand{\officiumtitulum}[1]{
%  \newpage
%  \thispagestyle{empty}
  \setheaders{{{\scshape\addfontfeature{LetterSpace=3.0}}Infra Octavam Corporis Christæ.}}{{{\scshape\addfontfeature{LetterSpace=3.0} #1}}}
 \section{
  \begin{center}
  {\large#1}
  \end{center}
}}

\newcommand{\subtitulum}[1]{
 \subsection{
  \begin{center}
  {\large#1}
  \end{center}
}}

\begin{document}

\thispagestyle{empty}
\greannotation{Tonus solemnior}
\greannotation{℣.}
\grechangestyle{initial}{\fontsize{32}{32}\selectfont}
\gregorioscore{Toni_Communes/Deus_in_adjutorium_tonus_solemnior}
\vspace{2mm}
\rubrique{vel}
\greannotation{Tonus festivus}
\greannotation{℣.}
\gregorioscore{Toni_Communes/Deus_In_Adjutorium_Festivus}

\officiumtitulum{
In I. Vesperis.}
\vspace{-0.5cm}

\grechangedim{maxbaroffsettextleft}{0cm}{fixed}
\greannotation{1. Ant.}
\greannotation{1. f}
\grechangestyle{initial}{\fontsize{28}{28}\selectfont}
\gregorioscore{Sacerdos_in_aeternum/Sacerdos_in_aeternum}
\begin{center}
Psalmus 109.
\end{center}
\vspace{-3mm}
\smallscore{109_1f}
\pstext{109_mode_1}

\grechangedim{maxbaroffsettextleft}{0cm}{fixed}
\greannotation{2. Ant.}
\greannotation{2. D}
\grechangestyle{initial}{\fontsize{28}{28}\selectfont}
\gregorioscore{Miserator_dominus/Miserator_dominus}
\begin{center}
Psalmus 110.
\end{center}
\vspace{-3mm}
\smallscore{110_2}
\pstext{110_mode_2}

\greannotation{3. Ant.}
\greannotation{3. a\textsuperscript{2}}
\grechangestyle{initial}{\fontsize{28}{28}\selectfont}
\gregorioscore{Calicem_salutaris/Calicem_salutaris}
\begin{center}
Psalmus 115.
\end{center}
\vspace{-3mm}
\smallscore{115_3a2}
\pstext{115_mode_3}
%
%
\greannotation{4. Ant.}
\greannotation{4. E}
\grechangestyle{initial}{\fontsize{28}{28}\selectfont}
\gregorioscore{Sicut_novellae/Sicut_novellae}
\begin{center}
Psalmus 127.
\end{center}
\vspace{-3mm}
\smallscore{127_4E}
\pstext{127_mode_4}
%
\grechangedim{maxbaroffsettextleft}{0.1cm}{fixed}
\grechangedim{maxbaroffsettextright}{0.15cm}{fixed}
\greannotation{5. Ant.}
\greannotation{5. a}
\grechangestyle{initial}{\fontsize{28}{28}\selectfont}
\gregorioscore{Qui_pacem_ponit/Qui_pacem_ponit}
\begin{center}
Psalmus 147.
\end{center}
\vspace{-3mm}
\smallscore{147_5}
\pstext{147_mode_5}
%

\begin{center}
Capitulum.
\vspace{-3mm}
\end{center}

\Capitulum{Capitulum_CC}

\begin{center}
Hymnus.
\vspace{-3mm}
\end{center}
%
\greannotation{3.}
\gregorioscore{Pange_Lingua/Pange_lingua_mode_3}
\vscore{Versus_in_festo_Corporis_Christi}
\vspace{-0.5cm}
\rubrique{(Sic cantatur in I. et II. Vesperis Festi tantum; alias in tono communi.)}
\vspace{2mm}

\grechangedim{annotationseparation}{0.03cm}{fixed}
\greannotation{Ad Magnif.}
\greannotation{Ant. 6. F}
\grechangestyle{initial}{\fontsize{28}{28}\selectfont}
\gregorioscore{O_quam_suavis/O_quam_suavis}

\begin{center}
Oratio.
\vspace{-3mm}
\end{center}

Orémus.\\
Deus, qui nobis sub Sacraménto mirábili passiónis tuæ memóriam reliquísti: † tríbue, quǽsumus, ita nos córporis, et sánguinis tui sacra mystéria venerári; * ut redemptiónis tuæ fructum in nobis júgiter sentiámus. Qui vivis et regnas cum Deo Patre in unitate.
\vspace{0.5cm}

\greannotation{1.}
\grechangestyle{initial}{\fontsize{28}{28}\selectfont}
\gregorioscore{Toni_Communes/Benedicamus_Domino_I_Vesp}
%
\officiumtitulum{
In II. Vesperis.}
\vspace{-0.5cm}

\rubrique{Omnia sicut in I. Vesperis.}

\grechangedim{annotationseparation}{-0.01cm}{fixed}
\greannotation{Ad Magnif.}
\greannotation{Ant. 5.}
\grechangestyle{initial}{\fontsize{28}{28}\selectfont}
\gregorioscore{O_sacrum_convivium/O_sacrum_convivium}

\grechangestyle{initial}{\fontsize{28}{28}\selectfont}
\gregorioscore{Toni_Communes/Benedicamus_Domino_Solemnis_mode_6}
\medskip
\greannotation{5.}
\grechangestyle{initial}{\fontsize{28}{28}\selectfont}
\gregorioscore{Toni_Communes/Benedicamus_Domino_Solemnis_mode_5}

\subtitulum{
Sabbato ad Vesperas.}
\vspace{-0.5cm}

\rubrique{Antiphonæ et Psalmi ut in I. Vesperis Festi.}

\rubrique{Capitulum. \normaltext{Carissime: Nolite.} ut supra ad Vesperas Dominicis. Hymnus \normaltext{Pange Lingua., 7}}

℣. Cibávit iílos ex ádippe fruménti, allelúia.\\
\hspace*{6mm}℟. Et de pétra, mélle saturávit éos, allelúia.
\vspace{2mm}

\grechangedim{annotationseparation}{-0.01cm}{fixed}
\greannotation{Ad Magnif.}
\greannotation{Ant. 7. a}
\grechangestyle{initial}{\fontsize{28}{28}\selectfont}
\gregorioscore{Puer_samuel/Puer_samuel}
\vspace{2mm}

\rubrique{Oratio ut supra ad Vesperas Dominicis.}

\rubrique{Et fit commemoratio de Octava, Ant. \normaltext{O sacrum convivium. ℣. Panem de cælo.} oratio \normaltext{Deus qui nobis.}}

\officiumtitulum{Dominica infra octavam Corporis Christi.}
\vspace{-0.9cm}
\begin{center}\addfontfeature{LetterSpace=3.0}{\textsc{quæ est II. post Pentecosten.}}\end{center}
%{\centering{\addfontfeature{LetterSpace=3.0}{\textsc{quæ est II. post Pentecosten.}}}
\vspace{-0.1cm}
\begin{center}
Ad Vesperas.
\end{center}

\rubrique{Antiphonæ et Psalmi ut in I. Vesperis Festi.}

\begin{center}
Capitulum.
\vspace{-3mm}
\end{center}

\Capitulum{Capitulum_Dominicis}
\vspace{2mm}

\rubrique{Hymnus. \normaltext{Pangue lingua., 7.}}

℣. Cibávit iílos ex ádippe fruménti, allelúia.\\
\hspace*{6mm}℟. Et de pétra, mélle saturávit éos, allelúia.
\vspace{2mm}

\grechangedim{annotationseparation}{-0.01cm}{fixed}
\greannotation{Ad Magnif.}
\greannotation{Ant. 1. a}
\grechangestyle{initial}{\fontsize{28}{28}\selectfont}
\gregorioscore{Exi_cito/Exi_cito}

\begin{center}
Oratio.
\vspace{-3mm}
\end{center}

Orémus.\\
Sancti nóminis tui, Dómine, timórem páriter et amórem fac nos habére perpétuum: † quia nunquam tua gubernatióne destítuis,* quos in soliditáte tuæ dilectiónis instítuis.
Per Dóminum.
%\vspace{0.5cm}

\rubrique{Deinde fit commemoratio de Octava cum Antiphona et ℣. ut in I. Vesperis.}

\rubrique{Si tamen sequenti die faciendum non sit Officium de Octava, dicitur Ant.
\normaltext{O sacrum convivium.} ut in II. Vesperis Festi.}

%\vspace{8pt}
\greannotation{2.}
\grechangestyle{initial}{\fontsize{28}{28}\selectfont}
\gregorioscore{Toni_Communes/Benedicamus_domino_Semiduplex_Vesp}
\vspace{2mm}
Feria Quarta. \rubrique{Ad Vesperas, omnia ut in I. Vesperis Festi.}

\rubrique{Si tamen sequenti die occurrat Festum Nativtatis S. Joannis Baptistae vel SS. Apostolorum Petri et Pauli in I. Vesperis pro commemoratione præcedentis diei infra Octavam sumitur Antiphona \normaltext{O sacrum convivium.} ut in II. Vesperis Festi.}

\enddocument