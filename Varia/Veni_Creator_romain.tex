% !TEX TS-program = lualatexmk
% !TEX parameter =  --shell-escape
\documentclass[11pt]{book}
\usepackage[a5paper]{geometry}
%\pagenumbering{gobble}
\usepackage{fontspec}
\usepackage[ecclesiasticlatin.usej]{babel}
     \babelprovide[hyphenrules=latin]{ecclesiasticlatin}
          \usepackage{lettrine}
\usepackage[autocompile]{gregoriotex}
\def\GreStar{*}
\def\GreDagger{†}
%\gresetheadercapture{annotation}{greannotation}{string}
%
\AtBeginDocument{\setlength{\parindent}{1em}} 

\setmainfont{EB Garamond}[UprightFont=EB Garamond Regular,
ItalicFont= EB Garamond Italic,
BoldFont= EB Garamond Bold,
Ligatures=Rare,
Numbers=Proportional,
Numbers=OldStyle]

\NewDocumentCommand{\rubrique}{m}{{\fontsize{8}{10}\selectfont\textit{#1}}}

\begin{document}

\thispagestyle{empty}

\rubrique{The celebrant intones, and the cantors continue. The people may join the schola on even verses.}

\rubrique{Vowels in italics are not pronounced; the consonant is pronounced with the following vowel as in v. 4, or the usual melody is sung, which is the case in v. 6.}

\grechangestyle{initial}{\fontsize{28}{28}\selectfont}
\greannotation{Hy.}
\greannotation{8.}
\gregorioscore{partitions/hy_veni_creator_spiritus_solesmes}

℣. Emítte Spíritum tuum, et creabúntur.

℟. Et renovábis fáciem terræ.

Orémus.

\lettrine{D}{e}us, qui corda fidélium Sancti Spíritus illustratióne docuísti:~† da nobis in eódem Spíritu re\-cta sápere,~* et de ejus semper consolatióne gaudére. Per Christum Dóminum nostrum. ℟. Amen.

\end{document}